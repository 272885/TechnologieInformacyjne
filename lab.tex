\documentclass{article}
\usepackage[utf8]{inputenc}
\usepackage[polish]{babel}
\usepackage[T1]{fontenc}






\title{Spadek swobodny}
\author{Mateusz Głuchowski}
\date{December 2022}

\begin{document}

\maketitle

\section{Wprowadzenie teoretyczne}
\begin{flushleft}
    Spadek swobodny jest definiowany jako ruch odbywający się pod wpływem ciężaru bez oporów ośrodka.
\end{flushleft}

\begin{flushleft}
    
    Jeżeli spadek spełnia warunki takie jak:\\
        \> •	ma miejsce z małej wysokości w pobliżu Ziemi\\
        \> •	dotyczy ciała o stosunkowo dużej gęstości\\
        \> •	ciało posiada aerodynamiczny kształt,\\

\end{flushleft}

\begin{flushleft}
    wówczas można ruch takiego ciała traktować z przybliżeniem jako ruch jednostajnie przyspieszony z przyspieszeniem ziemskim nieposiadającym prędkości początkowej. W tym przypadku ruch ten opisuje kinetyczne równianie ruchu:

    \begin{equation}
        h_{(t)} = h_{0} - \frac{g*t^{2}}{2}
    \end{equation}

    gdzie: \\
    \> $h_{(t)}$ - wysokość na jakiej znajduje się ciało po czasie t \\ 
    \> $h_{0}$ - początkowa wysokość ciała (wysokość z jakiej spada ciało)\\
    \> $t$ - czas spadania ciała \\
    \> $g$ - przyspieszenie ziemskie ($9,81 \frac{m}{s^{2}}$)\\
    
\end{flushleft}

\begin{flushleft}
    W spadku swobodnym czas spadania ciała oznacza się wzorem:
    \begin{equation}
    t = \sqrt{\frac{2*h}{g}}
    \end{equation}

    gdzie: \\
    \> $t$ - czas spadania ciała\\
    \> $h$ - wysokość\\
    \> $g$ - przyspieszenie ziemskie ($9.81 \frac{m}{s^{2}}$) \\
    
\end{flushleft}

\section{Przebieg eskperymentu}

\begin{figure}
    \includegraphics[1]{}
\end{figure}

\section{Wyniki pomiarów}

\section{Wnioski}

\end{document}
